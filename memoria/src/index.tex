\documentclass{article}
\usepackage{graphicx}
\graphicspath{{src/img/}}
\begin{document}

\begin{titlepage}
\centering
\includegraphics[width=0.35\textwidth]{logo-uned.jpeg}

\vspace{1.5cm}

{\Large Universidad Nacional de Educación a Distancia\par}
\vspace{0.4cm}
{\large Escuela técnica superior de ingeniería informática\par}

\vspace{1.6cm}

{\large Proyecto de Fin de Grado en Ingeniería Informática\par}

\vspace{1.6cm}

{\LARGE \textbf{Aplicación web de nutrición}\par}

\vspace{2.2cm}

{\large María de la Asunción Gómez Castro\par}

\vfill

{\large Curso 2025-2026\par}
\end{titlepage}

\section*{Resumen}
\addcontentsline{toc}{section}{Resumen}
El acceso a información nutricional y la adopción de dietas equilibradas representan un reto cotidiano para muchas personas. Mantener una alimentación variada y saludable, que además considere ingredientes de cercanía y temporada, implica gestionar distintos factores: el valor nutricional de cada alimento, la planificación de menús, y la generación de listas de la compra adecuadas. La dificultad de coordinar todos estos aspectos suele traducirse en hábitos alimentarios poco organizados y, a menudo, menos saludables de lo deseado.

Por otra parte, hoy disponemos de multitud de recursos tecnológicos que pueden facilitar este proceso. Existen bases de datos de alimentos con información detallada, herramientas para digitalizar productos mediante códigos de barras, y una amplia oferta de recetas en línea. Los recientes avances en inteligencia artificial permiten, además, procesar y analizar información de manera eficiente y personalizada.

Este proyecto surge de la necesidad de unificar todas esas herramientas en una única plataforma que simplifique la gestión nutricional y la organización de la cocina diaria. El enfoque consiste en integrar capacidades de importación de recetas, análisis nutricional, planificación de menús y automatización de listas de la compra, haciendo uso de fuentes de datos existentes y de tecnologías como inteligencia artificial, en una aplicación de fácil acceso para usuarios y profesionales de la nutrición.

\newpage

\section*{Abstract}
\addcontentsline{toc}{section}{Abstract}
Accessing nutritional information and adopting balanced diets is an everyday challenge for many people. Maintaining a varied and healthy diet, while also considering local and seasonal ingredients, involves managing various factors: the nutritional value of each food, menu planning, and generating appropriate shopping lists. The complexity of coordinating all these aspects often results in poorly organized eating habits that are less healthy than intended.

On the other hand, we now have a multitude of technological resources that can help in this process. There are food databases with detailed information, tools for digitizing products using barcode readers, and a wide range of recipes available online. Recent advances in artificial intelligence also enable efficient and personalized processing and analysis of information.

This project was created out of the need to unify all these tools in a single platform that simplifies nutritional management and daily kitchen organization. The approach consists of integrating recipe import capabilities, nutritional analysis, menu planning, and shopping list automation—leveraging existing data sources and technologies like artificial intelligence—into an application that is easily accessible for both individual users and nutrition professionals.

\newpage
\renewcommand{\contentsname}{Índice}
\addcontentsline{toc}{section}{Índice}
\tableofcontents

\end{document}