\section*{Anexo I: Historias de usuario}
\phantomsection
\addcontentsline{toc}{section}{Anexo I: Historias de usuario}
\label{sec:anexo-historias-usuario}

Las historias de usuario constituyen una técnica fundamental en el desarrollo ágil de software para capturar requerimientos desde la perspectiva del usuario final. Según Cohn \cite{Cohn2004}, "una historia de usuario describe funcionalidad que será valiosa para un usuario o cliente de un sistema o software", y están compuestas por tres aspectos: una descripción escrita utilizada para planificación y como recordatorio, conversaciones sobre la historia que ayudan a elaborar los detalles, y pruebas que pueden usarse para determinar cuándo una historia está completa1. Estas historias sirven como herramienta para alcanzar tanto requerimientos funcionales como no funcionales, funcionando como pasos individuales e incrementales que van acercando el desarrollo hacia el objetivo final del proyecto.

Cada historia de usuario representa una funcionalidad específica que aporta valor al usuario y puede ser implementada de forma independiente en una iteración del proyecto. La asignación de estas historias a las diferentes iteraciones se realizará en función de las prioridades establecidas a los requerimientos con los que se asocian, asegurando que las funcionalidades más críticas se desarrollen en primer lugar.

Para facilitar la planificación y estimación del esfuerzo de desarrollo, cada historia de usuario será valorada utilizando puntos de historia basados en la sucesión de Fibonacci (1, 2, 3, 5, 8). Como explica Cohn \cite{Cohn2005}, los estudios demuestran que somos mejores estimando cosas que caen dentro de un orden de magnitud, y estas secuencias no lineales funcionan bien porque reflejan la mayor incertidumbre asociada con estimaciones para unidades de trabajo más grandes. Los espacios en la secuencia de Fibonacci se vuelven apropiadamente más grandes a medida que los números aumentan, donde cada número debe pensarse como un "cajón" en el que se vierten elementos del tamaño apropiado. Los puntos no representan tiempo directamente, sino una medida relativa de complejidad, esfuerzo y riesgo asociado a cada historia.

Cada historia de usuario desarrollada estará explícitamente relacionada con uno o varios requerimientos funcionales o no funcionales identificados previamente, estableciendo una trazabilidad clara entre las necesidades documentadas y su implementación práctica. Esta relación garantiza que cada incremento de funcionalidad contribuya de manera medible al cumplimiento de los objetivos del proyecto.

\subsection*{Configuración inicial}
\addcontentsline{toc}{subsection}{Configuración inicial}

\us[
	identificador=US-CI-01,
	titulo={Creación del repositorio de código},
	como={Desarrollador},
	quiero={Controlar las versiones de mi código, compartirlo con otros compañeros y mantener copias de seguridad para mejorar el acceso al mismo},
	requerimientos={\reqref{RNF-CA-01}, \reqref{RNF-DE-02}},
	puntos={1},
	criterios={
			\begin{itemize}
				\item El repositorio de código en la nube existe.
				\item Es accesible por todos los desarrolladores.
				\item Incluye instrucciones de uso.
			\end{itemize}
		},
	tareas={
			\begin{itemize}
				\item Inicialización de Git en un directorio local.
				\item Creación de un repositorio en Github.
				\item Configuración de acceso remoto al repositorio de Github desde la máquina local.
				\item Redacción de las instrucciones de uso en un README.
			\end{itemize}
		}
]

\us[
	identificador=US-CI-02,
	titulo={Creación de una SPA usando Vite y React},
	como={Desarrollador},
	quiero={Ver mi proyecto en funcionamiento para poder verificar que los cambios que realizo tienen el efecto deseado},
	requerimientos={\reqref{RNF-CO-01}, \reqref{RNF-DE-04}, \reqref{RNF-RE-01}},
	puntos={1},
	criterios={
			\begin{itemize}
				\item Existe una aplicación Vite + React.
				\item La aplicación puede ejecutarse localmente.
				\item La aplicación puede visitarse en el navegador.
				\item La aplicación responde de forma automática a los cambios llevados a cabo en el código fuente.
			\end{itemize}
		},
	tareas={
			\begin{itemize}
				\item Creación de un proyecto Vite + React.
				\item Personalización de la configuración.
			\end{itemize}
		}
]

\us[
	identificador=US-CI-03,
	titulo={Controles de calidad con ESLint y Prettier},
	como={Desarrollador},
	quiero={Asegurar la calidad del código y mantener un estilo de codificación consistente},
	requerimientos={\reqref{RNF-CA-01}, \reqref{RNF-CA-02}},
	puntos={3},
	criterios={
			\begin{itemize}
				\item Existe una Github Action que ejecuta ESLint y Prettier sobre el código fuente.
				\item El repositorio sólo permite cambios en la rama main mediante Pull Request.
				\item La Github Action se ejecuta automáticamente cuando se crea un Pull Request.
				\item La Pull Request no se puede fusionar hasta que la Github Action haya finalizado correctamente.
				\item Existen comandos documentados para ejecutar ESLint y Prettier sobre el código fuente en local.
			\end{itemize}
		},
	tareas={
			\begin{itemize}
				\item Instalación de ESLint y Prettier en el proyecto.
				\item Configuración de ESLint y Prettier.
				\item Configuración de Github Actions para ejecutar ESLint y Prettier.
				\item Documentación de los comandos para ejecutar ESLint y Prettier en local.
				\item Configuración del repositorio de Github para que sólo se permitan cambios en la rama main mediante Pull Request.
				\item Configuración del repositorio para hacer la Github Action obligatoria para fusionar un Pull Request.
			\end{itemize}
		}
]

\us[
	identificador=US-CI-04,
	titulo={Corrección automática del código},
	como={Desarrollador},
	quiero={Tener correctores automáticos que modifican mi código para adaptarse a los estándares establecidos},
	requerimientos={\reqref{RNF-CA-02}},
	puntos={5},
	criterios={
			\begin{itemize}
				\item Existe un pre-commit hook que ejecuta ESLint y Prettier sobre el código fuente, modificando el código fuente para que cumpla con los estándares establecidos.
				\item El pre-commit hook sólo se ejecuta sobre los archivos que han sido modificados en el commit.
				\item El pre-commit hook corrige automáticamente los errores de formato y buenas prácticas de código.
				\item El pre-commit hook muestra un informe del resultado de su ejecución.
				\item Existe documentación clara sobre el uso del pre-commit hook.
			\end{itemize}
		},
	tareas={
			\begin{itemize}
				\item Encontrar un gestor de pre-commit hooks para el proyecto, teniendo en cuenta que se trata de un monorepo que tendrá sistemas en diferentes lenguajes de programación.
				\item Instalar el gestor de pre-commit hooks elegido.
				\item Añadir los correctores de ESLint y Prettier.
				\item Asegurarse de que los correctores sólo se ejecutan sobre ficheros del front-end.
				\item Asegurarse de que los correctores sólo se ejecutan sobre los ficheros que han sido modificados en el commit.
				\item Asegurarse de que el pre-commit hook corrige automáticamente los errores de formato y buenas prácticas de código.
				\item Asegurarse de que el pre-commit hook muestra un informe del resultado de su ejecución.
				\item Documentación de los comandos para ejecutar ESLint y Prettier en local.
			\end{itemize}
		}
]

\us[
	identificador=US-CI-05,
	titulo={Control de ortografía y gramática},
	como={Desarrollador},
	quiero={Tener control de ortografía y gramática en los textos de la aplicación},
	requerimientos={\reqref{RNF-CA-04}},
	puntos={5},
	criterios={
			\begin{itemize}
				\item El pre-commit hook ejecuta un análisis de ortografía y gramática sobre los textos visibles de la aplicación.
			\end{itemize}
		},
	tareas={
			\begin{itemize}
				\item Seleccionar un servicio de análisis de ortografía y gramática.
				\item Configurar el servicio de análisis de ortografía y gramática para que se ejecute en el pre-commit hook.
				\item Documentar el uso del servicio de análisis de ortografía y gramática en el pre-commit hook.
			\end{itemize}
		}
]

\subsection*{Despliegue en la nube}
\addcontentsline{toc}{subsection}{Despliegue en la nube}

\us[
	identificador=US-DE-01,
	titulo={Despliegue en la nube del entorno de desarollo},
	como={Desarrollador},
	quiero={Poder acceder a la aplicación desde cualquier dispositivo con acceso a internet},
	requerimientos={\reqref{RNF-DE-01}, \reqref{RNF-DE-02}},
	puntos={8},
	criterios={
			\begin{itemize}
				\item Existe un entorno de desarrollo en la nube.
				\item Cada nuevo commit a la rama main se despliega automáticamente a la nube.
				\item La aplicación es accesible a través de una URL de Internet.
			\end{itemize}
		},
	tareas={
			\begin{itemize}
				\item Selección de un proveedor de nube para el entorno de desarrollo.
				\item Diseño de la infraestructura en la nube necesaria para el entorno.
				\item Configuración de la infraestructura como código.
				\item Creación de la infraestructura en el ciclo de CD (Continuous Deployment).
				\item Despliegue del código fuente del front-end a la nube.
			\end{itemize}
		}
]

\us[
	identificador=US-DE-02,
	titulo={Despliegue en la nube del entorno de producción},
	como={Usuario},
	quiero={Poder acceder a la aplicación desde cualquier dispositivo con acceso a internet},
	requerimientos={\reqref{RNF-CO-01}, \reqref{RNF-DE-01}, \reqref{RNF-DE-02}, \reqref{RNF-DE-03}, \reqref{RNF-RE-04}},
	puntos={8},
	criterios={
			\begin{itemize}
				\item Existe un entorno de producción en la nube.
				\item El despliegue en producción se realiza manualmente, sin automatización.
				\item La aplicación es accesible a través de una URL de Internet.
				\item La infraestructura de la aplicación en producción es idéntica a la del entorno de desarrollo.
			\end{itemize}
		},
	tareas={
			\begin{itemize}
				\item Creación de la infraestructura de producción a partir de la infraestructura de desarrollo.
				\item Configuración de despliegue manual para producción.
			\end{itemize}
		}
]

\subsection*{Maquetación del front-end}
\addcontentsline{toc}{subsection}{Maquetación del front-end}

\us[
	identificador=US-MF-01,
	titulo={Maquetación de la pantalla de inicio},
	como={Usuario},
	quiero={Entender la funcionalidad de la aplicación desde el momento del acceso},
	requerimientos={\reqref{RF-GU-01}, \reqref{RF-GU-02}, \reqref{RF-IR-07}, \reqref{RNF-CO-01}, \reqref{RNF-CO-02}},
	puntos={8},
	criterios={
			\begin{itemize}
				\item La pantalla de inicio tiene una barra de navegación.
				\item La barra de navegación presenta el logo y un acceso directo a la pantalla de inicio.
				\item La barra de inicio presenta opciones para registrarse o iniciar sesión.
				\item La pantalla de inicio tiene una hero section.
				\item La hero section describe claramente el propósito de la aplicación.
				\item La hero section presenta un \textit{call to action} que guía el viaje de usuario hacia el registro o el inicio de sesión.
				\item La pantalla de inicio presenta una lista de recetas destacadas.
				\item La pantalla de inicio presenta un pie de página con información de contacto y enlaces a los documentos legales.
			\end{itemize}
		},
	tareas={
			\begin{itemize}
				\item Selección de la paleta de colores, tipografías y elementos gráficos que representarán la marca \textbf{Muncher} en todas las pantallas.
				\item Selección de tecnologías front-end a utilizar: librería de componentes (preexistente o propia), sistema de gestión de estilos (CSS puro, CSS Modules, CSS in JS o Tailwind CSS).
				\item Maquetación de la pantalla de inicio de forma responsiva y accesible.
				\item Implementación de la pantalla de inicio en el front-end.
			\end{itemize}
		}
]

\us[
	identificador=US-MF-02,
	titulo={Internacionalización},
	como={Usuario},
	quiero={Acceder al contenido en mi idioma preferido},
	requerimientos={\reqref{RF-GU-01}, \reqref{RF-GU-02}, \reqref{RF-IR-07}, \reqref{RNF-CO-01}, \reqref{RNF-CO-02}},
	puntos={8},
	criterios={
			\begin{itemize}
				\item Los textos de la aplicación se almacenan de forma separada al código.
				\item Los textos de la aplicación se encuentran en varios idiomas.
				\item El usuario puede seleccionar el idioma de la aplicación.
				\item La aplicación detecta el idioma del navegador del usuario y lo utiliza como idioma por defecto.
			\end{itemize}
		},
	tareas={
			\begin{itemize}
				\item Implementación de la internacionalización en el front-end.
			\end{itemize}
		}
]