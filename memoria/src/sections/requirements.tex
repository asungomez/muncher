\section*{Requerimientos}
\addcontentsline{toc}{section}{Requerimientos}

Los requerimientos de software constituyen la base fundamental para el desarrollo exitoso de cualquier sistema informático. Según Sommerville \cite{Sommerville2016}, los requerimientos describen los servicios que debe proporcionar el sistema y las restricciones bajo las cuales debe operar, estableciendo un contrato entre desarrolladores y usuarios sobre lo que el sistema debe hacer. Pressman y Maxim \cite{PressmanMaxim2020} complementan esta definición señalando que los requerimientos representan una condición o capacidad necesaria para que un usuario resuelva un problema o alcance un objetivo específico.

La ingeniería de requerimientos es un proceso crítico que, según el IEEE Standard 830-1998 \cite{IEEE830-1998}, debe resultar en especificaciones que sean correctas, no ambiguas, completas, consistentes, verificables, modificables y trazables. Una gestión inadecuada de los requerimientos es una de las principales causas de fracaso en proyectos de software, como demuestran múltiples estudios del Standish Group en sus informes CHAOS \cite{StandishGroupCHAOS}.

Los requerimientos se clasifican tradicionalmente en dos categorías principales: requerimientos funcionales, que describen lo que el sistema debe hacer, y requerimientos no funcionales, que especifican cómo debe comportarse el sistema en términos de calidad, rendimiento y restricciones (Wiegers y Beatty \cite{Wiegers2013}).

En esta sección se presentan los requerimientos específicos para \textbf{Muncher}, clasificados según su naturaleza y priorizados de acuerdo con su importancia para el éxito del proyecto. Esta especificación detallada servirá como guía para el diseño, implementación y validación del sistema.

