\section*{Requerimientos}
\addcontentsline{toc}{section}{Requerimientos}

Los requerimientos de software constituyen la base fundamental para el desarrollo exitoso de cualquier sistema informático. Según Sommerville \cite{Sommerville2010}, los requerimientos describen los servicios que debe proporcionar el sistema y las restricciones bajo las cuales debe operar, reflejando las necesidades de los clientes.

La ingeniería de requerimientos es un proceso crítico que, según el IEEE Standard 830-1998 \cite{IEEE830-1998}, debe resultar en especificaciones que sean correctas, no ambiguas, completas, consistentes, ordenadas por importancia, verificables, modificables y trazables. Como indica el Project Management Institute \cite{PMI2021}, una gestión ineficaz de los requerimientos puede llevar a retrabajos, expansión del alcance, insatisfacción del cliente, sobrecostes, retrasos en el cronograma y, en general, al fracaso del proyecto.

Los requerimientos se clasifican tradicionalmente en dos categorías principales: requerimientos funcionales, que describen el comportamiento observable del sistema bajo diversas circunstancias, y requerimientos no funcionales, que especifican características importantes del sistema como disponibilidad, usabilidad, seguridad o rendimiento (Wiegers y Beatty \cite{Wiegers2013}).

En esta sección se presentan los requerimientos específicos para \textbf{Muncher}, clasificados según su naturaleza y priorizados de acuerdo con su importancia para el éxito del proyecto. Esta especificación detallada servirá como guía para el diseño, implementación y validación del sistema.

\subsection*{Priorización de requerimientos}
\addcontentsline{toc}{subsection}{Priorización de requerimientos}

La priorización de requerimientos es un proceso fundamental en el desarrollo de software que consiste en ordenar los requerimientos según su importancia relativa para el éxito del proyecto. Como señalan Karlsson y Ryan \cite{Karlsson1997}, "un conocimiento claro e inequívoco sobre las prioridades de los requerimientos ayuda a enfocar el proceso de desarrollo y gestionar los proyectos de manera más efectiva y eficiente".

La importancia de una priorización efectiva radica en que no todos los requerimientos tienen el mismo impacto en el valor del producto final. Una priorización adecuada permite tomar decisiones informadas sobre qué características implementar primero, facilita la gestión de cambios durante el desarrollo y ayuda a establecer una hoja de ruta clara para las entregas del proyecto.

\subsubsection*{Método MoSCoW}
\addcontentsline{toc}{subsubsection}{Método MoSCoW}

Para la priorización de los requerimientos de Muncher se utilizará el método MoSCoW, una técnica de priorización desarrollada como parte del framework DSDM (Dynamic Systems Development Method) \cite{DSDM2014} que ayuda a comprender y gestionar prioridades de manera efectiva.

Según el DSDM Consortium, en proyectos donde el tiempo es un factor limitante, es vital comprender la importancia relativa del trabajo a realizar para mantener el progreso y cumplir con los plazos establecidos. El método MoSCoW clasifica los requerimientos en cuatro categorías claramente definidas:

\begin{itemize}
    \item \textbf{Must Have (M):} Los requerimientos Must Have constituyen el Subconjunto Mínimo Utilizable (MUST) que el proyecto garantiza entregar, siendo absolutamente esenciales para que la solución sea viable, legal o segura. La prueba decisiva para identificarlos es preguntarse: "¿Qué sucede si este requerimiento no se cumple?"; si la respuesta implica cancelar el proyecto porque no tendría sentido implementar la solución sin él, entonces es un Must Have, mientras que si existe alguna alternativa (aunque sea manual y laboriosa), pertenece a categorías de menor prioridad.
    \item \textbf{Should Have (S):} Los requerimientos Should Have se definen como importantes pero no vitales: puede ser doloroso omitirlos, pero la solución sigue siendo viable. Pueden requerir algún tipo de solución alternativa como gestión de expectativas, cierta ineficiencia, una solución existente o trabajo manual, que puede ser temporal. La diferencia clave entre un Should Have y un Could Have radica en el grado de dolor causado por no cumplir el requerimiento, medido en términos de valor de negocio o número de personas afectadas.
    \item \textbf{Could Have (C):} Los requerimientos Could Have se definen como deseables pero menos importantes, con menor impacto si se omiten comparado con los Should Have. Estos requerimientos proporcionan la principal reserva de contingencia del proyecto, ya que solo se entregarían en su totalidad en el mejor escenario posible. Cuando surge un problema y la fecha límite está en riesgo, los Could Have constituyen la primera opción de elementos a eliminar del marco temporal actual.
    \item \textbf{Won't Have This Time (W):} Los requerimientos Won't Have This Time son aquellos que el equipo del proyecto ha acordado que no serán entregados como parte de este marco temporal. Se registran en la Lista de Requerimientos Priorizados donde ayudan a clarificar el alcance del proyecto, evitando que sean reintroducidos informalmente en una fecha posterior. Esto ayuda a gestionar las expectativas de que algunos requerimientos simplemente no llegarán a la Solución Desplegada, al menos no en esta ocasión.
\end{itemize}



\subsection*{Requerimientos funcionales}
\addcontentsline{toc}{subsection}{Requerimientos funcionales}

Los requerimientos funcionales especifican las funcionalidades concretas que debe proporcionar el sistema para satisfacer las necesidades identificadas de los usuarios. Estos requerimientos describen las capacidades específicas, comportamientos y servicios que \textbf{Muncher} debe ofrecer, estableciendo qué debe hacer el sistema sin entrar en detalles sobre cómo debe hacerlo.

La especificación detallada de los requerimientos funcionales es fundamental para garantizar que el sistema desarrollado cumpla con las expectativas de los usuarios y proporcione valor real en la resolución de los problemas identificados. Para facilitar su gestión y trazabilidad a lo largo del proyecto, los requerimientos se han organizado en categorías temáticas coherentes con los objetivos funcionales previamente establecidos:

\begin{itemize}
    \item \textbf{Gestión de usuarios (GU):} Requerimientos relacionados con la gestión de usuarios y su acceso al sistema.
    \item \textbf{Introducción de recetas (IR):} Requerimientos relacionados con la introducción de recetas en el sistema.
    \item \textbf{Planificación de menús (PM):} Requerimientos relacionados con la planificación de menús.
    \item \textbf{Consulta de valores nutricionales (VN):} Requerimientos relacionados con la consulta de valores nutricionales.
    \item \textbf{Gestión de listas de la compra (LC):} Requerimientos relacionados con la gestión de listas de la compra.
\end{itemize}

Cada requerimiento funcional ha sido identificado con un código único siguiendo el formato \textbf{RF-XX-YY}, donde \textbf{RF} indica "Requerimiento Funcional", \textbf{XX} indica el tipo de requerimiento funcional y \textbf{YY} corresponde a un número secuencial. Esta codificación permite la trazabilidad completa desde la especificación inicial hasta la implementación, pruebas y validación final, facilitando la gestión de cambios y el seguimiento del progreso del desarrollo.



\subsubsection*{Gestión de usuarios}
\addcontentsline{toc}{subsubsection}{Gestión de usuarios}

\req[
  identificador=RF-GU-01,
  titulo={Registro de usuarios},
  descripcion={El sistema permitirá a los usuarios registrarse en la aplicación.},
  prioridad={Must Have}
]

\req[
  identificador=RF-GU-02,
  titulo={Inicio de sesión},
  descripcion={El sistema permitirá a los usuarios iniciar sesión en la aplicación.},
  prioridad={Must Have}
]

\req[
  identificador=RF-GU-03,
  titulo={Cierre de sesión},
  descripcion={El sistema permitirá a los usuarios cerrar sesión en la aplicación.},
  prioridad={Must Have}
]

\req[
  identificador=RF-GU-04,
  titulo={Administración de usuarios},
  descripcion={El sistema permitirá a los usuarios administradores gestionar los usuarios de la aplicación.},
  prioridad={Should Have}
]

\req[
  identificador=RF-GU-05,
  titulo={Eliminación de cuentas de usuario},
  descripcion={El sistema permitirá a los usuarios administradores eliminar cuentas de usuario.},
  prioridad={Won't Have This Time}
]

\subsubsection*{Introducción de recetas}
\addcontentsline{toc}{subsubsection}{Introducción de recetas}

\req[
  identificador=RF-IR-01,
  titulo={Introducción manual de recetas},
  descripcion={El sistema permitirá a los usuarios introducir recetas manualmente.},
  prioridad={Must Have}
]

\req[
  identificador=RF-IR-02,
  titulo={Análisis automático de recetas},
  descripcion={El sistema analizará automáticamente recetas de fuentes externas y las incorporará al inventario del usuario.},
  prioridad={Should Have}
]

\req[
  identificador=RF-IR-03,
  titulo={Clasificación de recetas},
  descripcion={El sistema permitirá a los usuarios clasificar las recetas en categorías.},
  prioridad={Could Have}
]

\req[
  identificador=RF-IR-04,
  titulo={Búsqueda de recetas},
  descripcion={El sistema permitirá a los usuarios buscar recetas por nombre, ingredientes o categorías.},
  prioridad={Must Have}
]

\req[
  identificador=RF-IR-05,
  titulo={Edición de recetas},
  descripcion={El sistema permitirá a los usuarios editar las recetas.},
  prioridad={Must Have}
]

\req[
  identificador=RF-IR-06,
  titulo={Eliminación de recetas},
  descripcion={El sistema permitirá a los usuarios eliminar recetas.},
  prioridad={Must Have}
]

\req[
  identificador=RF-IR-07,
  titulo={Visualización de recetas},
  descripcion={El sistema permitirá a los usuarios visualizar las recetas.},
  prioridad={Must Have}
]

\req[
  identificador=RF-IR-08,
  titulo={Creación de variantes de recetas},
  descripcion={El sistema permitirá a los usuarios crear variantes de recetas existentes para adaptarlas a diferentes escenarios.},
  prioridad={Could Have}
]

\req[
  identificador=RF-IR-09,
  titulo={Enriquecimiento de recetas},
  descripcion={El sistema enriquecerá las recetas con información adicional como las dietas en las que se puede incluir o las precauciones de los ingredientes.},
  prioridad={Could Have}
]

\subsubsection*{Planificación de menús}
\addcontentsline{toc}{subsubsection}{Planificación de menús}

\req[
  identificador=RF-PM-01,
  titulo={Planificación manual de menú},
  descripcion={El sistema permitirá a los usuarios planificar menús añadiendo recetas manualmente.},
  prioridad={Must Have}
]

\req[
  identificador=RF-PM-02,
  titulo={Planificación automática de menú},
  descripcion={El sistema permitirá a los usuarios generar un menú planificado automáticamente en base a una serie de parámetros: ingredientes disponibles, restricciones alimentarias, objetivos, restricciones de tiempo, etc.},
  prioridad={Could Have}
]

\req[
  identificador=RF-PM-03,
  titulo={Visualización de menús},
  descripcion={El sistema permitirá a los usuarios visualizar los menús planificados.},
  prioridad={Must Have}
]

\req[
  identificador=RF-PM-04,
  titulo={Edición de menús},
  descripcion={El sistema permitirá a los usuarios editar los menús planificados.},
  prioridad={Must Have}
]

\req[
  identificador=RF-PM-05,
  titulo={Eliminación de menús},
  descripcion={El sistema permitirá a los usuarios eliminar los menús planificados.},
  prioridad={Must Have}
]

\subsubsection*{Consulta de valores nutricionales}
\addcontentsline{toc}{subsubsection}{Consulta de valores nutricionales}

\req[
  identificador=RF-VN-01,
  titulo={Consulta de valores nutricionales de una receta},
  descripcion={El sistema permitirá a los usuarios consultar los valores nutricionales de una receta.},
  prioridad={Must Have}
]

\req[
  identificador=RF-VN-02,
  titulo={Consulta de valores nutricionales de un periodo de tiempo},
  descripcion={El sistema permitirá a los usuarios consultar los valores nutricionales de un periodo de tiempo, como un día o una semana.},
  prioridad={Must Have}
]

\req[
  identificador=RF-VN-03,
  titulo={Establecimiento manual de objetivos nutricionales},
  descripcion={El sistema permitirá a los usuarios establecer objetivos nutricionales manualmente.},
  prioridad={Should Have}
]

\req[
  identificador=RF-VN-04,
  titulo={Establecimiento de objetivos nutricionales automáticos},
  descripcion={El sistema permitirá a los usuarios establecer objetivos nutricionales automáticamente en base a su peso, altura, edad, objetivos, etc.},
  prioridad={Could Have}
]

\req[
  identificador=RF-VN-05,
  titulo={Visualización de progreso en la consecución de objetivos nutricionales},
  descripcion={El sistema permitirá a los usuarios visualizar el progreso en la consecución de sus objetivos nutricionales.},
  prioridad={Should Have}
]

\subsubsection*{Gestión de listas de la compra}
\addcontentsline{toc}{subsubsection}{Gestión de listas de la compra}

\req[
  identificador=RF-LC-01,
  titulo={Creación manual de listas de la compra},
  descripcion={El sistema permitirá a los usuarios crear listas de la compra manualmente.},
  prioridad={Must Have}
]

\req[
  identificador=RF-LC-02,
  titulo={Creación automática de listas de la compra},
  descripcion={El sistema permitirá a los usuarios crear listas de la compra automáticamente en base a los ingredientes necesarios para las recetas planificadas.},
  prioridad={Must Have}
]

\req[
  identificador=RF-LC-03,
  titulo={Visualización de listas de la compra},
  descripcion={El sistema permitirá a los usuarios visualizar las listas de la compra.},
  prioridad={Must Have}
]

\req[
  identificador=RF-LC-04,
  titulo={Edición de listas de la compra},
  descripcion={El sistema permitirá a los usuarios editar las listas de la compra.},
  prioridad={Must Have}
]

\req[
  identificador=RF-LC-05,
  titulo={Eliminación de listas de la compra},
  descripcion={El sistema permitirá a los usuarios eliminar las listas de la compra.},
  prioridad={Must Have}
]

\req[
  identificador=RF-LC-06,
  titulo={Progreso de listas de la compra},
  descripcion={El sistema permitirá a los usuarios marcar los ingredientes ya disponibles o comprados en la lista de la compra a medida que se van comprando.},
  prioridad={Should Have}
]

\subsection*{Requerimientos no funcionales}
\addcontentsline{toc}{subsection}{Requerimientos no funcionales}

Los requerimientos no funcionales especifican las cualidades y restricciones que debe cumplir el sistema en términos de rendimiento, usabilidad, seguridad y otros atributos de calidad. A diferencia de los requerimientos funcionales, que describen qué debe hacer el sistema, los requerimientos no funcionales establecen cómo debe comportarse y bajo qué condiciones debe operar para proporcionar una experiencia satisfactoria a los usuarios.

Estos requerimientos son fundamentales para el éxito del proyecto, ya que determinan aspectos críticos como la facilidad de uso, la confiabilidad, la seguridad y el rendimiento del sistema. Su correcta especificación y cumplimiento puede marcar la diferencia entre una aplicación técnicamente funcional y una verdaderamente exitosa en el mercado.

Para facilitar su organización y trazabilidad, los requerimientos no funcionales se han estructurado en las siguientes categorías, alineadas con los objetivos no funcionales previamente establecidos:

\begin{itemize}
    \item \textbf{Compatibilidad (CO):} Requerimientos relacionados con la utilización en diferentes dispositivos.
    \item \textbf{Inteligencia Artificial (IA):} Requerimientos sobre la integración de herramientas de IA.
    \item \textbf{Calidad (CA):} Requerimientos de automatización de controles de calidad.
    \item \textbf{Despliegue (DE):} Requerimientos de automatización del despliegue.
    \item \textbf{Monitorización (MO):} Requerimientos de seguimiento del estado del sistema.
    \item \textbf{Seguridad (SE):} Requerimientos de protección y control de acceso.
    \item \textbf{Rendimiento (RE):} Requerimientos de velocidad y eficiencia.
    \item \textbf{Usabilidad (US):} Requerimientos de facilidad de uso y experiencia de usuario.
\end{itemize}

Cada requerimiento no funcional ha sido identificado con un código único siguiendo el formato \textbf{RNF-XX-YY}, donde RNF indica "Requerimiento No Funcional", XX corresponde al código de categoría y YY a un número secuencial dentro de esa categoría.

\subsubsection*{Compatibilidad}
\addcontentsline{toc}{subsubsection}{Compatibilidad}

\req[
  identificador=RNF-CO-01,
  titulo={Tecnologías web},
  descripcion={El sistema utilizará tecnologías web estándar, que facilitarán su acceso desde cualquier dispositivo con navegación web.},
  prioridad={Must Have}
]

\req[
  identificador=RNF-CO-02,
  titulo={Responsividad},
  descripcion={El sistema será responsivo, adaptándose a diferentes tamaños de pantalla y resoluciones.},
  prioridad={Must Have}
]

\subsubsection*{Inteligencia Artificial}
\addcontentsline{toc}{subsubsection}{Inteligencia Artificial}

\req[
  identificador=RNF-IA-01,
  titulo={Introducción automática de datos},
  descripcion={El sistema utilizará tecnologías de IA para introducir datos de forma automática a partir de imágenes o texto en lenguaje natural.},
  prioridad={Must Have}
]

\req[
  identificador=RNF-IA-02,
  titulo={Generación automática de sugerencias},
  descripcion={El sistema utilizará tecnologías de IA para generar sugerencias de recetas, menús, ingredientes, etc.},
  prioridad={Could Have}
]

\subsubsection*{Calidad}
\addcontentsline{toc}{subsubsection}{Calidad}

\req[
  identificador=RNF-CA-01,
  titulo={Integración continua},
  descripcion={El sistema tendrá integración continua, para validar la correcta funcionalidad del mismo.},
  prioridad={Must Have}
]

\req[
  identificador=RNF-CA-02,
  titulo={Control de calidad de código},
  descripcion={El sistema tendrá un control de calidad de formato y buenas prácticas de código fuente, que se usará para validar los cambios antes de que se integren en el repositorio.},
  prioridad={Must Have}
]

\req[
  identificador=RNF-CA-03,
  titulo={Control de calidad de funcionalidad},
  descripcion={El sistema tendrá un control de calidad de funcionalidad, que se usará para validar que el sistema cumple con los requerimientos funcionales.},
  prioridad={Must Have}
]

\req[
  identificador=RNF-CA-04,
  titulo={Análisis de gramática y ortografía},
  descripcion={El sistema tendrá un análisis de gramática y ortografía, que se usará para validar que el texto introducido es correcto.},
  prioridad={Should Have}
]

\subsubsection*{Despliegue}
\addcontentsline{toc}{subsubsection}{Despliegue}

\req[
  identificador=RNF-DE-01,
  titulo={Despliegue en la nube},
  descripcion={El sistema se desplegará en la nube, para poder acceder a él desde cualquier dispositivo con acceso a internet.},
  prioridad={Must Have}
]

\req[
  identificador=RNF-DE-02,
  titulo={Despliegue continuo},
  descripcion={Cada nuevo cambio en el código fuente se desplegará a la nube de forma automática.},
  prioridad={Must Have}
]

\req[
  identificador=RNF-DE-03,
  titulo={Separación de entornos},
  descripcion={Existirán dos entornos de nube diferentes: uno para desarrollo y otro para producción.},
  prioridad={Could Have}
]

\req[
  identificador=RNF-DE-04,
  titulo={Entorno local},
  descripcion={El proyecto incluirá mecanismos e instrucciones para levantar localmente un entorno idéntico a la nube que el desarrollador pueda usar para probar el sistema antes de enviar sus cambios en el código.},
  prioridad={Must Have}
]

\subsubsection*{Monitorización}
\addcontentsline{toc}{subsubsection}{Monitorización}

\req[
  identificador=RNF-MO-01,
  titulo={Monitorización de errores},
  descripcion={El sistema tendrá un sistema de monitorización de errores, para poder detectar y solucionar los errores que se produzcan.},
  prioridad={Must Have}
]

\req[
  identificador=RNF-MO-02,
  titulo={Alertas ante errores},
  descripcion={El sistema tendrá un sistema de alertas ante errores, para poder notificar a los desarrolladores cuando se produzcan errores.},
  prioridad={Should Have}
]

\req[
  identificador=RNF-MO-03,
  titulo={Monitorización de rendimiento},
  descripcion={El sistema tendrá un sistema de monitorización de rendimiento, para poder detectar y solucionar los problemas de rendimiento que se produzcan. Incluirá trazas que informen, para cada operación, del tiempo invertido en ella.},
  prioridad={Could Have}
]

\req[
  identificador=RNF-MO-04,
  titulo={Monitorización de tráfico},
  descripcion={El sistema tendrá un sistema de monitorización de tráfico, para conocer en todo momento el número de usuarios que están usando el sistema.},
  prioridad={Could Have}
]

\subsubsection*{Seguridad}
\addcontentsline{toc}{subsubsection}{Seguridad}

\req[
  identificador=RNF-SE-01,
  titulo={JSON Web Tokens},
  descripcion={El sistema utilizará JSON web tokens para verficar la identidad de los usuarios.},
  prioridad={Must Have}
]

\req[
  identificador=RNF-SE-02,
  titulo={Encriptación},
  descripcion={El sistema encriptará las credenciales del usuario antes de almacenarlas entre sesiones.},
  prioridad={Must Have}
]

\req[
  identificador=RNF-SE-03,
  titulo={Almacenamiento seguro de secretos},
  descripcion={Los datos sensibles del sistema, como las claves de acceso a la nube o la contraseña de la base de datos, se almacenarán de forma segura.},
  prioridad={Must Have}
]

\req[
  identificador=RNF-SE-04,
  titulo={Monitorización de vulnerabilidades},
  descripcion={El sistema tendrá un sistema de monitorización de vulnerabilidades, para poder detectar y solucionar las vulnerabilidades que se produzcan.},
  prioridad={Should Have}
]

\subsubsection*{Rendimiento}
\addcontentsline{toc}{subsubsection}{Rendimiento}

\req[
  identificador=RNF-RE-01,
  titulo={Uso de SPA},
  descripcion={La interfaz de usuario se implementará como una SPA (Single Page Application), para asegurar que el feedback a las interacciones es inmediato.},
  prioridad={Must Have}
]

\req[
  identificador=RNF-RE-02,
  titulo={Caché de datos en el front-end},
  descripcion={El front-end del sistema tendrá una caché interna de los datos recibidos del back-end, para ahorrar tiempo en llamadas sucesivas.},
  prioridad={Must Have}
]

\req[
  identificador=RNF-RE-03,
  titulo={Servicio de índice de datos},
  descripcion={El sistema tendrá un servicio de índice de datos sobre la base de datos, para optimizar el tiempo de consultas.},
  prioridad={Should Have}
]

\req[
  identificador=RNF-RE-04,
  titulo={Red de distribución de contenido},
  descripcion={Los ficheros estáticos de la aplicación, como imágenes o vídeos, se distribuirán mediante una CDN (Content Delivery Network), para reducir la latencia de las peticiones.},
  prioridad={Could Have}
]

\subsubsection*{Usabilidad}
\addcontentsline{toc}{subsubsection}{Usabilidad}

\req[
  identificador=RNF-US-01,
  titulo={Diseño minimalista},
  descripcion={El sistema tendrá un diseño minimalista, para que se pueda usar sin necesidad de instrucciones.},
  prioridad={Must Have}
]

\req[
  identificador=RNF-US-02,
  titulo={Validación de formularios},
  descripcion={Los formularios incluirán validación en tiempo real con mensajes de error claros y específicos.},
  prioridad={Should Have}
]

\req[
  identificador=RNF-US-03,
  titulo={Legibilidad},
  descripcion={Los textos tendrán suficiente contraste de color para garantizar la legibilidad.},
  prioridad={Must Have}
]
