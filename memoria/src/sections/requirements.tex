\section*{Requerimientos}
\addcontentsline{toc}{section}{Requerimientos}

Los requerimientos de software constituyen la base fundamental para el desarrollo exitoso de cualquier sistema informático. Según Sommerville \cite{Sommerville2010}, los requerimientos describen los servicios que debe proporcionar el sistema y las restricciones bajo las cuales debe operar, reflejando las necesidades de los clientes.

La ingeniería de requerimientos es un proceso crítico que, según el IEEE Standard 830-1998 \cite{IEEE830-1998}, debe resultar en especificaciones que sean correctas, no ambiguas, completas, consistentes, ordenadas por importancia, verificables, modificables y trazables. Como indica el Project Management Institute \cite{PMI2021}, una gestión ineficaz de los requerimientos puede llevar a retrabajos, expansión del alcance, insatisfacción del cliente, sobrecostes, retrasos en el cronograma y, en general, al fracaso del proyecto.

Los requerimientos se clasifican tradicionalmente en dos categorías principales: requerimientos funcionales, que describen el comportamiento observable del sistema bajo diversas circunstancias, y requerimientos no funcionales, que especifican características importantes del sistema como disponibilidad, usabilidad, seguridad o rendimiento (Wiegers y Beatty \cite{Wiegers2013}).

En esta sección se presentan los requerimientos específicos para \textbf{Muncher}, clasificados según su naturaleza y priorizados de acuerdo con su importancia para el éxito del proyecto. Esta especificación detallada servirá como guía para el diseño, implementación y validación del sistema.

