\section*{Objetivos}
\addcontentsline{toc}{section}{Objetivos}

Esta sección establece los objetivos que guían tanto la elaboración de este documento como el desarrollo de la aplicación \textbf{Muncher}. Se estructura en dos partes diferenciadas pero complementarias.

En primer lugar, se definen los \textbf{objetivos del documento}, que especifican el propósito y alcance de esta memoria técnica, describiendo qué aspectos del proyecto se documentan y con qué nivel de detalle. Estos objetivos orientan la estructura y contenido del presente trabajo, asegurando que se cubran todas las fases relevantes del desarrollo del sistema.

En segundo lugar, se presentan los \textbf{objetivos de la aplicación}, divididos en objetivos funcionales y no funcionales. Los objetivos funcionales describen las capacidades y características que debe ofrecer \textbf{Muncher} a sus usuarios, es decir, qué problemas específicos debe resolver y qué funcionalidades debe proporcionar. Los objetivos no funcionales establecen los criterios de calidad, rendimiento, usabilidad y otros aspectos técnicos que el sistema debe cumplir para garantizar una experiencia satisfactoria y un funcionamiento robusto.

Juntos, estos objetivos proporcionan una hoja de ruta clara tanto para la documentación del proyecto como para su implementación técnica, estableciendo las metas que se pretenden alcanzar y los criterios que permitirán evaluar el éxito del desarrollo.

\subsection*{Objetivos del documento}
\addcontentsline{toc}{subsection}{Objetivos del documento}

El presente documento tiene como finalidad proporcionar una documentación completa y estructurada del proyecto \textbf{Muncher}, abarcando desde la conceptualización inicial hasta la implementación y validación del sistema. Los objetivos específicos que guían la elaboración de esta memoria son los siguientes:

\paragraph*{Establecer una planificación sistemática del proyecto}
mediante la definición clara de los requerimientos funcionales y no funcionales que debe cumplir la aplicación. Esta planificación incluye la clasificación de dichos requerimientos por orden de prioridad, lo que permitirá guiar el desarrollo de manera eficiente y asegurar que las funcionalidades más críticas se implementen en primer lugar.

\paragraph*{Definir los casos de uso más generales}
que representan las interacciones principales entre los usuarios y el sistema. A partir de estos casos de uso, se desarrollará una lista detallada de historias de usuario que permita completar cada funcionalidad de manera incremental y orientada a las necesidades reales de los usuarios finales.

\paragraph*{Documentar la arquitectura del sistema}
de forma justificada, explicando las decisiones de diseño adoptadas y su impacto en el rendimiento, escalabilidad y mantenibilidad de la aplicación. Esta documentación incluirá también la selección y justificación de las herramientas tecnológicas utilizadas en cada capa del sistema.

\paragraph*{Describir el modelo de datos}
necesario para soportar todas las funcionalidades previstas, detallando las entidades, relaciones y restricciones que permitan una implementación robusta y eficiente de la base de datos.

\paragraph*{Explicar las pruebas del sistema}
requeridas para garantizar la funcionalidad, fiabilidad y calidad de la aplicación, incluyendo estrategias de testing que validen tanto los aspectos funcionales como no funcionales del sistema.

\paragraph*{Documentar el proceso de despliegue del sistema en la nube}
explicando las decisiones tomadas respecto a la infraestructura, servicios utilizados, configuración del entorno de producción y estrategias de monitorización. Esta documentación incluirá la justificación de las opciones seleccionadas en términos de escalabilidad, coste y mantenimiento.

\paragraph*{Presentar las conclusiones del proyecto}
evaluando el grado de cumplimiento de los objetivos planteados, analizando las lecciones aprendidas durante el desarrollo y identificando posibles mejoras o líneas de trabajo futuro que puedan enriquecer la aplicación.

Este enfoque metodológico asegura que el documento sirva como guía técnica completa para el desarrollo, implementación y mantenimiento futuro de \textbf{Muncher}.