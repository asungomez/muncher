\section*{Objetivos}
\addcontentsline{toc}{section}{Objetivos}

Esta sección establece los objetivos que guían tanto la elaboración de este documento como el desarrollo de la aplicación \textbf{Muncher}. Se estructura en dos partes diferenciadas pero complementarias.

En primer lugar, se definen los \textbf{objetivos del documento}, que especifican el propósito y alcance de esta memoria técnica, describiendo qué aspectos del proyecto se documentan y con qué nivel de detalle. Estos objetivos orientan la estructura y contenido del presente trabajo, asegurando que se cubran todas las fases relevantes del desarrollo del sistema.

En segundo lugar, se presentan los \textbf{objetivos de la aplicación}, divididos en objetivos funcionales y no funcionales. Los objetivos funcionales describen las capacidades y características que debe ofrecer \textbf{Muncher} a sus usuarios, es decir, qué problemas específicos debe resolver y qué funcionalidades debe proporcionar. Los objetivos no funcionales establecen los criterios de calidad, rendimiento, usabilidad y otros aspectos técnicos que el sistema debe cumplir para garantizar una experiencia satisfactoria y un funcionamiento robusto.

Juntos, estos objetivos proporcionan una hoja de ruta clara tanto para la documentación del proyecto como para su implementación técnica, estableciendo las metas que se pretenden alcanzar y los criterios que permitirán evaluar el éxito del desarrollo.