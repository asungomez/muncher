% Descripción de un requerimiento
% Ejemplo de uso:
% \req[
%   identificador=RNF-RE-04,
%   titulo={Red de distribución de contenido},
%   descripcion={Los ficheros estáticos...},
%   prioridad=Could Have
% ]
\ExplSyntaxOn
\keys_define:nn { req }
  {
    identificador .tl_set:N = \l__req_id_tl,
    titulo .tl_set:N = \l__req_title_tl,
    descripcion .tl_set:N = \l__req_description_tl,
    prioridad .tl_set:N = \l__req_priority_tl
  }
\NewDocumentCommand{\req}{ O{} }
  {
    \keys_set:nn { req } { #1 }
    {\small
    \par\vspace{0.8\baselineskip}\noindent\phantomsection\hypertarget{req:\l__req_id_tl}{}\label{req:\l__req_id_tl}%
    \textbf{\l__req_title_tl}\par\vspace{0.3\baselineskip}%
    \noindent\begin{tabular}{@{}>{\bfseries\RaggedRight\arraybackslash}p{4.0cm}p{\dimexpr\linewidth-4.0cm-2\tabcolsep\relax}@{}}
    \arrayrulecolor{lightgray}
    Identificador: & \texttt{\l__req_id_tl} \\
    \hline
    Descripción: & {\l__req_description_tl} \\
    \hline
    Prioridad~(MoSCoW): & {\l__req_priority_tl} \\
    \end{tabular}\par
    }% end small
    \vspace{0.6\baselineskip}%
  }
\ExplSyntaxOff

% Referencia a un requerimiento por su identificador con un enlace clicable
% Ejemplo de uso: \reqref{RNF-SE-01}
\newcommand{\reqref}[1]{\hyperref[req:#1]{\texttt{#1}}}

% Historia de usuario: identificador, título, rol (como), objetivo (quiero), puntos, criterios, tareas
% Ejemplo de uso:
% \us[
%   identificador=US-XX-YY,
%   titulo={Mi historia},
%   como={Rol},
%   quiero={Objetivo},
%   puntos={3},
%   criterios={...},
%   tareas={...}
% ]
\ExplSyntaxOn
\keys_define:nn { us }
  {
    identificador .tl_set:N = \l__us_id_tl,
    titulo .tl_set:N = \l__us_title_tl,
    como .tl_set:N = \l__us_role_tl,
    quiero .tl_set:N = \l__us_goal_tl,
    requerimientos .tl_set:N = \l__us_requirements_tl,
    puntos .tl_set:N = \l__us_points_tl,
    criterios .tl_set:N = \l__us_acceptance_tl,
    tareas .tl_set:N = \l__us_tasks_tl
  }
\NewDocumentCommand{\us}{ O{} }
  {
    \keys_set:nn { us } { #1 }
    \par\vspace{0.8\baselineskip}\noindent\phantomsection\hypertarget{us:\l__us_id_tl}{}\label{us:\l__us_id_tl}%
    \textbf{\l__us_title_tl}\par\vspace{0.3\baselineskip}%
    \noindent\begin{tabular}{@{}>{\bfseries\RaggedRight\arraybackslash}p{4.0cm}p{\dimexpr\linewidth-4.0cm-2\tabcolsep\relax}@{}}
    \arrayrulecolor{lightgray}
    Identificador: & \texttt{\l__us_id_tl} \\
    \hline
    Como: & {\l__us_role_tl} \\
    \hline
    Quiero: & {\l__us_goal_tl} \\
    \hline
    Requerimientos: & {\l__us_requirements_tl} \\
    \hline
    Puntos~de~historia: & {\l__us_points_tl} \\
    \hline
    Criterios~de~aceptación: & {\l__us_acceptance_tl} \\
    \hline
    Tareas: & {\l__us_tasks_tl} \\
    \end{tabular}\par
  }
\ExplSyntaxOff


% Referencia a una historia de usuario por su identificador con un enlace clicable
% Ejemplo de uso: \usref{US-XX-YY}
\newcommand{\usref}[1]{\hyperref[us:#1]{\texttt{#1}}}

